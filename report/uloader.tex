\newcommand{\titel}{Abschlussbericht ARM}
\title{\titel}
\newcommand{\untertitel}{Abschlussbericht ARM   2014/2015}
%\subtitle{\untertitel}
\newcommand{\autor}{Pedram Mehrikki(MATRIKELNR)\\Appel, Dennis (813783)\\Voigt, Alexander (814526)}
\author{\autor}
\date{\today}

\documentclass[
	11pt,				% Schriftgröße
	DIV10,
	german,				% für Umlaute, Silbentrennung etc.
	a4paper,			% Papierformat
	oneside,			% einseitiges Dokument
	titlepage,			% es wird eine Titelseite verwendet
	parskip=half,			% Abstand zwischen Absätzen (halbe Zeile)
	headings=normal,		% Größe der Überschriften verkleinern
	listof=totoc,			% Verzeichnisse im Inhaltsverzeichnis aufführen
	bibliography=totoc,		% Literaturverzeichnis im Inhaltsverzeichnis aufführen
	index=totoc,			% Index im Inhaltsverzeichnis aufführen
	captions=tableheading,		% Beschriftung von Tabellen unterhalb ausgeben
	final,				% Status des Dokuments (final/draft)
	numbers=endperiod
]{scrreprt}

\usepackage{scrhack}
\usepackage{yfonts}
% Zum Umfließen von Bildern -------------------------------------------------
\usepackage{floatflt}
\usepackage{lmodern}
\usepackage{textcomp}			% Euro-Zeichen etc.
\usepackage{listings}
\lstset{ %
	captionpos=b
}
\usepackage[ngerman]{babel}		% neue deutsche Rechtschreibung
\usepackage[utf8]{inputenc}
\usepackage[T1]{fontenc}
\usepackage{url}			% URL-Highlighting
\usepackage{amsmath,amsfonts}
% \usepackage{amsmath}
%\usepackage{times}
\usepackage{setspace}
\usepackage{multirow}
%\usepackage{amsfonts}
%\usepackage{amssymb}
\usepackage{array}
\usepackage[
	automark,			% Kapitelangaben in Kopfzeile automatisch erstellen
	headsepline,			% Trennlinie unter Kopfzeile
	ilines				% Trennlinie linksbündig ausrichten
]{scrpage2}
\usepackage[labelfont=bf, belowskip=4pt, hypcap]{caption}
\usepackage{a4wide}
%\usepackage{mathrsfs}
%\usepackage{mathtools}
\usepackage{graphicx}
%\usepackage{wasysym}
%\usepackage{pgfplots}
\usepackage{cite}
% Text um Bilder fliessen lassen
\usepackage{wrapfig}
\usepackage{float}
\usepackage{pdfpages}
\usepackage{geometry}
% \geometry{paper=a4paper,left=35mm,right=35mm,top=30mm}
\usepackage[square]{natbib}
\usepackage{tikz}
\usetikzlibrary{arrows,shapes,positioning,shadows,trees}

\tikzset{
  basic/.style  = {draw, text width=2cm, drop shadow, font=\sffamily, rectangle},
  root/.style   = {basic, rounded corners=2pt, thin, align=center,
                   fill=green!30},
  level 2/.style = {basic, rounded corners=6pt, thin,align=center, fill=green!60,
                   text width=8em},
  level 3/.style = {basic, thin, align=left, fill=pink!60, text width=8em}
}

% Kopf- und Fußzeilen
\pagestyle{scrheadings}

% Kopf- und Fußzeile auch auf Kapitelanfangsseiten
\renewcommand*{\chapterpagestyle}{scrheadings}

% Schriftform der Kopfzeile
\renewcommand{\headfont}{
	\normalfont
}

% Kopfzeile
\ihead{\large{\textsc{\titel}}\\ \textit{\headmark}}
\chead{}
\ohead{} %\vspace{0.5cm} \includegraphics[scale=0.3]{Beuth_Logo_horizontal}}
%
\setlength{\headheight}{21mm}		% Höhe der Kopfzeile
\setlength{\parindent}{5mm}		% Einzug bei neuem Absatz
\setheadwidth[0pt]{textwithmarginpar}	% Kopfzeile über den Text hinaus verbreitern
\setheadsepline[text]{0.4pt}		% Trennlinie unter Kopfzeile

% Fußzeile
\ifoot{}
\cfoot{\textsc{\footnotesize{\untertitel}}}
\ofoot{\pagemark}

\subtitle{\untertitel}

% erzeugt ein wenig mehr Platz hinter einem Punkt
\frenchspacing

% Schusterjungen und Hurenkinder vermeiden
\clubpenalty		= 10000
\widowpenalty		= 10000
\displaywidowpenalty	= 10000

% mach pdf searchable
\usepackage{lmodern}
\input{glyphtounicode}
\pdfgentounicode=1

% \usepackage[bookmarks,bookmarksnumbered]{hyperref}
\usepackage[
bookmarks,
bookmarksopen=true,
pdftitle={\titel},
% pdfauthor={\autor},
% pdfcreator={\autor},
pdfsubject={\titel},
pdfkeywords={\titel},
colorlinks=true,
linkcolor=red, % einfache interne Verknüpfungen
anchorcolor=black,% Ankertext
citecolor=blue, % Verweise auf Literaturverzeichniseinträge im Text
filecolor=magenta, % Verknüpfungen, die lokale Dateien öffnen
menucolor=red, % Acrobat-Menüpunkte
urlcolor=cyan,
% linkcolor=black, % einfache interne Verknüpfungen
% anchorcolor=black,% Ankertext
% citecolor=black, % Verweise auf Literaturverzeichniseinträge im Text
% filecolor=black, % Verknüpfungen, die lokale Dateien öffnen
% menucolor=black, % Acrobat-Menüpunkte
% urlcolor=black,
backref,
%pagebackref,
plainpages=false,% zur korrekten Erstellung der Bookmarks
pdfpagelabels,% zur korrekten Erstellung der Bookmarks
hypertexnames=false,% zur korrekten Erstellung der Bookmarks
% linktocpage, % Seitenzahlen anstatt Text im Inhaltsverzeichnis verlinken
% linkcolor=red % Linkfarbe f\"ur den Druck auf schwarz, f\"ur die PDF-Version auf rot stellen
]{hyperref}

\setcounter{secnumdepth}{5}
\setcounter{tocdepth}{5}
\newcommand*{\fullref}[1]{\hyperref[{#1}]{\autoref*{#1} \nameref*{#1}}}

\makeindex

\begin{document}
\maketitle
\pagenumbering{Roman}
\tableofcontents
\listoffigures					% Abbildungsverzeichnis
\listoftables					% Tabellenverzeichnis
% \lstlistoflistings
\clearpage
\pagenumbering{arabic}
\chapter{Einleitung}
Das verwendete stm32f4-discovery Board ist ein beliebter "'Allrounder"'.
Die Begr\"undung daf\"ur ist das Preis-Leistungs-Verh\"altnis.
Daher ist dieses Board ein hervorragender Einstieg in die ARM-MCU-Programmierung. \\
Es gibt nahezu keinen Bereich, indem ARM-MCU heutzutage keine Verwendung finden. 
Das Projekt beinhaltet die Entwicklung eines Bootloaders sowie die Netzwerkimplementierung.
F\"ur die Lesbarkeit und Logik ist der Report in einzelne Bereiche aufgeteilt.

\chapter{Bootloader}
Ein Booloader ist ein Programm, welches es erm\"oglicht ein beliebiges weiteres
 Programm nachzuladen. Wobei das nachzuladene Programm das Eigentliche ist, 
welches ausgef\"uhrt werden soll.\\
Dazu ist notwendig die Hardware zu initialisieren, die ben\"otigt wird um das
 eigentliche Programm mit der gew\"ahlten Methode nachzuladen.\\
Das stm32f407 discovery board bietet drei verschiedene Arten an um zu booten.\\
Um zwischen den drei Varianten w\"ahlen zu k\"onnen, sind die Boot-Pins BOOT1
und BOOT2 entsprechend zu setzen:\\\\
\begin{tabular}{|c|c|l|l|}
\hline \hline
  BOOT1 & BOOT2 & Boot-Mode & Adresse\\ \hline
  x & 0 & Flash Memory (User Flash) & 0x8000\_0000\\
\hline
  0 & 1 & System Memory & 0x1FFF\_F000\\
\hline
   & 1 & SRAM & 0x2000\_0000\\
\hline
\end{tabular}\\\\
Der ROM, hier System Memory, beinhaltet den gegebenen Bootloader.\\
Wie man den Speicher dann belegt ist einem freigestellt. Man muss dann darauf
achten, dass man an die richtige Adresse springt, wenn man das Programm nachgeladen
hat.

\chapter{SWD}
Bei der Entwicklung kam die Serial Wire Debug Technologie zum Einsatz. Hierbei 
handelt es sich um einen Debug-Port, der speziell daf\"ur entwickelt wurde um MCU bzw. 
Projekte mit MCU, bei denen so wenig wie m\"oglich Pins verwendet werden sollen. \\
Dieser Port besteht aus zwei Leitungen:\\
\begin{itemize}
\item Clock
\item Bi-directional Data
\end{itemize}
Die Vorteile dieser Technologie sind (frei von der ARM-Website \"ubersetzt):
\begin{itemize}
\item Nur 2 Pins werden belegt
\item JTAG TAP controller kompatibel
\item Erlaubt dem Debugger ein weiter AMBA-Bus-Master zu werden um auf
Register / Speicher zuzugreifen.
\item High Datarates - 4Mbytes/sec @50MHz
\item Low Power - keine zus\"atzlichen Versorgungsspannung
\item gute "'built in"' Fehler-Erkennung
\item Schutz vor Fehlern bei Kontaktverlust
\end{itemize}  

\chapter{startup}
 - stack, program counter, interrupt, vector table, initial system clock

\chapter{CMSIS}
Der ARM Cortex Microcontroller Software Interface Standard ist eine 
H\"andlerunabh\"angige Abstraktionsschicht f\"ur die Cortex-M Prozessoren
 und beschreibt Debugger-Schnittstellen.

\chapter{NVIC}
Nested Vectored Interrupt Controller

\chapter{Unterschiede}
GNU, KEIL iar

\chapter{Netzwerk}
was sollte zum besseren verstehen hier einen platz finden.



\pagenumbering{roman}
\bibliography{Bibliographie}
\bibliographystyle{natdin}			% DIN-Stil des Literaturverzeichnisses
% Anhang -------------------------------------------------------------------
%		Die Inhalte des Anhangs werden analog zu den Kapiteln inkludiert.
%		Dies geschieht in der Datei Anhang.tex
% --------------------------------------------------------------------------
% \begin{appendix}
% 	\clearpage
% 	\input{appendix}
% \end{appendix}
\end{document}
