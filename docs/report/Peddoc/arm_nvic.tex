\chapter{NVIC in use}
As already said, the NVIC is used to configure the interrupts. Since
there is a bi directional communication via USART6 the NVIC has to be
configured to fit that demand.\\
Therefore the first thing to do is, to enable the clock on APB2 to clock
the USART6. Additonally configuration has to be done for the USARTX,
 but is not part of the NVIC configuration.\\
The clock is mentioned because it is an essential part to do that, neither 
U(S)ART[1..6] or whatever is used. Of course the the right APBX has to be
configured (precise diagram of stm32f4).\\
The configuration is done by writing the parameters into a structure and
bind and than load that struction into the NVIC.
The last thing to do is to enable USART globally.

\begin{lstlisting}%[ht]

/* We are initialized */
u->Initialized = 1;

/* Disable if not already */
USARTx->CR1 &= ~USART_CR1_UE;

/* Init */
USART_Init(USARTx, &USART_InitStruct);

/* Enable RX interrupt */
USARTx->CR1 |= USART_CR1_RXNEIE;

/* Fill NVIC settings */
NVIC_InitStruct.NVIC_IRQChannelCmd = ENABLE;
NVIC_InitStruct.NVIC_IRQChannelPreemptionPriority = TM_USART_NVIC_PRIORITY;
NVIC_InitStruct.NVIC_IRQChannelSubPriority = TM_USART_INT_GetSubPriority(USARTx);
NVIC_Init(&NVIC_InitStruct);

/* Enable USART peripheral */
USARTx->CR1 |= USART_CR1_UE;
\end{lstlisting}

As it is shown in the code the configurationprocess is done like described above.

The code is part of a function therefor USARTx is a parameter, in this case it
is USART6.
