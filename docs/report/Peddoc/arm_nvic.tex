\chapter{NVIC in use}
As already mentioned, the NVIC is used to configure the interrupts. Since
there is a bi-directional communication via USART6, the NVIC has to be
configured to fit that demand.\\
Therefore the first thing to do is to enable the clock on APB2 to clock
the USART6. Additonally configuration has to be done for the USARTX,
 but it is not part of the NVIC configuration.\\
The clock is mentioned because it plays an essential part of the configuration. 
It is not important which U(S)ART[1..6] pin is used. Naturally, the the right APBX 
has to be configured (precise diagram of stm32f4).\\
The configuration is done by writing the parameters into a structure and
binding them and than load it into the NVIC.
The last thing to do is to enable USART globally.

\begin{lstlisting}%[ht]

/* We are initialized */
u->Initialized = 1;

/* Disable if not already */
USARTx->CR1 &= ~USART_CR1_UE;

/* Init */
USART_Init(USARTx, &USART_InitStruct);

/* Enable RX interrupt */
USARTx->CR1 |= USART_CR1_RXNEIE;

/* Fill NVIC settings */
NVIC_InitStruct.NVIC_IRQChannelCmd = ENABLE;
NVIC_InitStruct.NVIC_IRQChannelPreemptionPriority = TM_USART_NVIC_PRIORITY;
NVIC_InitStruct.NVIC_IRQChannelSubPriority = TM_USART_INT_GetSubPriority(USARTx);
NVIC_Init(&NVIC_InitStruct);

/* Enable USART peripheral */
USARTx->CR1 |= USART_CR1_UE;
\end{lstlisting}

The code is part of a function therefor USARTx is a parameter, in this case it
is USART6.

The process and as well as the code that was used, generated the desired results whichs 
proves the method that was choosen works.


