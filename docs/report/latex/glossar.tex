\newglossaryentry{can}{name=CAN, description={Controller Area Network ist ein serrieller BUS der asynchron arbeitet. 1 Mbit/s ist hierbei die Maximale Datenrate. Wird meist in Fahrzeugen eingesetzt. }}
\newglossaryentry{ahb1}{name=AHB1, description={Advanced High-performance Bus is part of the Advanced Microcontroller Bus Architecture (AMBA) of the IP-manufacturer ARM Limited (ARM).}}
\newglossaryentry{apb}{name=APB, description={The Advanced Peripheral Bus (APB)
 is an internal bus for System-on-Chips (SoC) to connect low power peripheral
 devices. The APB bus is part of the AMBA-achitecture which is designed for low
power and simple interface. It can be used with the standardized buses like AHB.}}

\newglossaryentry{SWD}{name=SWD, description={Serial Wire Debug is an alternative 2-pin electrical interface that uses the same protocol. It uses the existing GND connection. SWD uses an ARM CPU standard bi-directional wire protocol, defined in the ARM Debug Interface v5.}}

\newglossaryentry{MCU}{name=MCU, description={Microcontroller Unit is a small computer on a single integrated circuit containing a processor core, memory, and programmable input/output peripherals. Program memory in the form of Ferroelectric RAM, NOR flash or OTP ROM is also often included on chip, as well as a typically small amount of RAM.}}

\newglossaryentry{LVM}{name=LVM, description={Logical Volume Manage provides a method of allocating space on mass-storage devices that is more flexible than conventional partitioning schemes. In particular, a volume manager can concatenate, stripe together or otherwise combine partitions (or block devices in general) into larger virtual ones that administrators can re-size or move, potentially without interrupting system use.}}

\newglossaryentry{ROM}{name=ROM, description={Read Only Memory is a class of storage medium used in computers and other electronic devices.}}

\newglossaryentry{USB}{name=USB, description={Universal Serial Bus is an industry standard developed in the mid-1990s that defines the cables, connectors and communications protocols used in a bus for connection, communication, and power supply between computers and electronic devices.}} 

\newglossaryentry{UART}{name=UART, description={universal asynchronous receiver/transmitter is a piece of computer hardware that translates data between parallel and serial forms. UARTs are commonly used in conjunction with communication standards such as EIA, RS-232, RS-422 or RS-485. The universal designation indicates that the data format and transmission speeds are configurable.}}

\newglossaryentry{JTAG}{name=JTAG, description={Joint Test Action Group was formed in 1985 to develop a method of testing finished printed circuit boards after manufacture. In 1990, the effort was codified as a standard by the Institute of Electrical and Electronics Engineers with the designation IEEE Std. 1149.1-1990 entitled Standard Test Access Port and Boundary-Scan Architecture.}}

\newglossaryentry{AMBA}{name=AMBA, description={Advanced Microcontroller Bus Architecture is an open-standard, on-chip interconnect specification for the connection and management of functional blocks in system-on-a-chip (SoC) designs. It facilitates development of multi-processor designs with large numbers of controllers and peripherals.}}

\newglossaryentry{CMSIS}{name=CMSIS, description={Microcontroller Software Interface Standard, is a vendor-independent hardware abstraction layer for the Cortex-M processor series and specifies debugger interfaces.}}

\newglossaryentry{AHB}{name=AHB, description={AHB, please refer to AMBA}}

\newglossaryentry{NVIC}{name=NVIC, description={Nested Vectored Interrupt Controller,facilitates low-latency exception and interrupt handling, controls power management, implements System Control Registers. The NVIC supports up to 240 dynamically reprioritizable interrupts each with up to 256 levels of priority.}}

\newglossaryentry{GPIO}{name=GPIO, description={General-purpose input/output is a generic pin on an integrated circuit whose behavior, including whether it is an input or output pin, can be controlled by the user at run time}}

\newglossaryentry{USART}{name=USART, description={universal synchronous/asynchronous receiver/transmitter, modern ICs now come with a UART that can also communicate synchronously}}

\newacronym{RCC}{RCC}{Reset and Clock Control}
\newacronym{TAP}{TAP}{test access ports}
\newacronym{AHB}{AHB}{Advanced High Performance Bus}
