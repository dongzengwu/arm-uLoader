\chapter{Implementation}
As mentioned earlier several makefiles are completely written from scratch.
There is a separate makefile for the Network Stack (lwip).
Also there is a config file (.config) that makes it possible to just compile in what is needed.
(And not everything and let the linker find what is needed, as used in IDE's.)

To save time the CMSIS libs and standard peripheral drivers supplied by ST have been used.
In addition the well written library from Tilen Majerle \citep{TM} have been used too.

To be more precisely the UART and therefor GPIO parts of them.
The advantage of them is that is now possible to make the input nonblocking,
so that an event is only fired when the buffer is either full or has a specific item in it.
(this corresponds to the canonical mode of \textit{termios} programming under Linux)

This behavior is important due the fact that the network stack needs occasionally time to
process pending work.

This corresponds well to the implementation idea of the lwip stack to be asynchron.
Lwip achieves that via callbacks.
That are registered functions that just get called when the specified event has been triggered.
Another advantage of that is that there is not the need to use an interrupt for every possible event.

Together those feature's make it possible to keep the Human Interface responsible
while processing pending work in the background.

The Human Interface makes it possible to interact with the User.
It can be compared to a Unix shell environment.
Corresponding to that it has some build in functions that are similar to some known from Unix.

A prompt showing the name of the Program awaits input.
This can be some calls like \textit{env} or \textit{printenv} that will show the
actual value the global variables have at the time of the call.
Some of them are the systemclock, IP and how many bytes the last file via TFTP was.

The call \textit{tftp} will issue an tftp read request to a hardcoded
server with a hardcoded file (blinky.bin).

But this will only work when there is a network connection and the device has acquired an IP address.
Appropriate to that the commands \textit{static} and \textit{dhcp} are available.
At system boot there are some attempts to get an IP from a DHCP server.
If that fails for some time an static IP will be set.
To try the DHCP process again of switch from dynamic to static IP allocation
these two command can be used at runtime.

The Trivial File Transfer Protocol \citep{tftp} was completely implemented by us,
via the RAW API from the lwip Stack.
As we don't have an multi threaded system the other two API's can not be used.
The netcon API needs an OS like Free RTOS and the Berkley socket API,
known from UNIX Systems, is build on top of that.

\section{Problems}
As there where other classes we need to do work for, the time was way to short
to implement all needed device drivers on our own, the help of the existing examples from
\textit{st} \citep{st-BB-examples} was therefore needed and welcome.

But the problem that has been arisen from the fact that they just use IDE's
specially build and pre configured for those examples (project file),
was so big and therefor time consuming,
that we could not manage to implement the main feature of a bootloader,
to load and start a downloaded program.

There is a command implemented to read the contents of the flash (\textit{read}).
It accepts an argument that stands for the number of bytes to read.
The output looks like the canonical ascii format known from \textit{hexdump}.

But a write command is still missing. Dues the fact that we could not find out in time
how we need to prepare the downloaded binary to make it start from the address we we would put it.
Guess it has to do something with the linker or startup script.

The most time consuming factor was to find out how to build the code and adjust the lwip stack
to work with the System. As the lwip wiki is kind of useless and not much documentation is available
for the now 3 year old last version of the stack.
To find out what to do, we were in need to reverse engineer the existing examples.

\section{Test}
The command:
\begin{verbatim}
	sudo picocom -b 38400 /dev/ttyUSB0 --omap crlf --echo
\end{verbatim}
was used, to get an interface via the uart6 of the Base Board and an ftdi adapter connected to the PC.
To verify that DHCP and TFTP work tcpdump was used.
