\chapter{Bootloader}
At first the bootloader is an unsolvable mystery but, how all unknown, after a
certain period of time keeping yourself busy with that subject, it loses all the
mystic glare and you recognize how easy it is.\\

\section{Preperation}
To enable a program to be written into the flash memory while the MCU is in
progress you have to do the things like setting up the peripherals needed for 
the process (descriped above).\\
After that is done the the necessary space, at the right place, has to be
allocated.

\begin{lstlisting}
void (*jumptoprogram)(void) = (void (*)(void))0x08008000;
\end{lstlisting}
The Address depends on the size of the Bootloader program, to make sure it won't
be overwritten.\\
Now the program has to be written to the address of the function pointer.
\begin{lstlisting}
unsigned int volatile * const program = (unsigned int *) 0x08008000;
\end{lstlisting}
Through the variable "program" it is possible to write to and read from
the specific address.
\begin{lstlisting}
*program = "'USART-Input"' /*write*/
\end{lstlisting}
Theoretically it is also possible to read from that address but that is not
used.
\section{Calling the Function}
At any point of the bootloader program it now can be jumped to the address
 by simply calling the function pointer.
\begin{lstlisting}
jumptoprogram();
\end{lstlisting}

From that point the bootloader is left behind and the "'real"' program is 
executing.
