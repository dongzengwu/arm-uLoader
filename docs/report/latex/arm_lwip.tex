\chapter{LWIP}
LwIP is open source TCP/IP protocol suite designed for embedded systems.\\
The focus of the LwIP TCP/IP implementation is to reduce RAM usage while still
having a full scale TCP. This makes LwIP
suitable for use in embedded systems with tens of kilobytes of free RAM and room
for around 40 kilobytes of code ROM.\\
LwIP includesteh following protocols and features:\citep{lwip-16}\\
\begin{itemize}
	\item IP (Internet Protocol) including packet forwarding over multiple network interfaces
  \item ICMP (Internet Control Message Protocol) for network maintenance and debugging
  \item IGMP (Internet Group Management Protocol) for multicast traffic management
  \item UDP (User Datagram Protocol) including experimental UDP-lite extensions
  \item TCP (Transmission Control Protocol) with congestion control, RTT estimation and fast recovery/fast retransmit
  \item Raw/native API for enhanced performance
  \item Optional Berkeley-like socket API
  \item DNS (Domain names resolver)
  \item SNMP (Simple Network Management Protocol)
  \item DHCP (Dynamic Host Configuration Protocol)
  \item AUTOIP (for IPv4, conform with RFC 3927)
  \item PPP (Point-to-Point Protocol)
  \item ARP (Address Resolution Protocol) for Ethernet 
\end{itemize}



